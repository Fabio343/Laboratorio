\documentclass[12pt]{article}
\usepackage{graphicx}
\usepackage{amsmath}
\usepackage{amssymb}
\usepackage{indentfirst}
\author{}
\title{}
\usepackage[paperwidth=600pt,paperheight=780pt,top=50pt,right=71pt,bottom=17pt,left=71pt]{geometry}
\usepackage[utf8]{inputenc}
\usepackage[portuguese]{babel}
\usepackage{float}


\makeatletter
	\newenvironment{indentation}[3]%
	{\par\setlength{\parindent}{#3}
	\setlength{\leftmargin}{#1}       \setlength{\rightmargin}{#2}%
	\advance\linewidth -\leftmargin       \advance\linewidth -\rightmargin%
	\advance\@totalleftmargin\leftmargin  \@setpar{{\@@par}}%
	\parshape 1\@totalleftmargin \linewidth\ignorespaces}{\par}%
\makeatother 
   
  
  \begin{document}
   \begin{figure}
   \centering
   \includegraphics[scale=0.5]{patrocinador-imeusp.png}
   {\tiny\caption{logo IME}}
   
   \end{figure}
  
   {\raggedright
   \begin{indentation}{2pt}{0pt}{0pt}
   \centering
   \textsf{POR QUE ESTUDAR  MAP 2212}
   
   \begin{figure}[!htb]
   \centering
   \includegraphics[scale=0.5]{duvidas.jpg}
   {\tiny\caption{Dúvida}}
  
   \end{figure}
   
   \end{indentation}
   }
   
   {\raggedright
   \begin{indentation}{300pt}{0pt}{0pt}
   \textsf{{\normalsize São Paulo, Março de 2018}}
   \end{indentation}
   }
   {\raggedright
   \begin{indentation}{0pt}{0pt}{0pt}
   \textsf{{\small Caro Julio Stern}}
   \end{indentation}
   }
   {\raggedright
   \begin{indentation}{0pt}{0pt}{0pt}
   \textsf{{ Professor do Departamento de Matemática Aplicada IME-USP}}
   \end{indentation}
   }
   
   \begin{indentation}{0pt}{0pt}{35pt} 
   \textsf{ O objetivo de está cursando uma disciplina não é de apenas  estar concluíndo os requisitos mínimos exigidos numa formação acadêmica, mas sim para adquirir novos métodos de análise e interpretação para possuir um destaque diferencial numa determinada área e/ou atividade profissional no futuro. Desse modo o motivo de se estudar  a disciplina Map 2212 é  para desenvolver as habilidades de programação, manipulação númerica e simulação de ambientes de aplicação estatistica, de forma  voltada a se obter conhecimento de mecânicas novas para serem aplicadas de diversas situações, como uso de softwares estatisticos, novas linguagens de programação, pois diversas áreas nescessitam de conhecimentos computacionais para realizar modelagem e analises, dentre outros, como por exemplo o LNCC(Laboratório nacional de computação científica) que realiza pesquisas em áreas interdisciplinares como bioinformática e biologia computacional.}
   
   \textsf{ Por isso tenho como objeto ao cursar essa disciplina de adquirir o máximo de habilidades em manipulação de problemas,  análises e desenvolvimento de estrategias além de tentar aprimorar as que já possuo ligadas a outros métodos computacionais como desenvolvimento web e ampliar as noções sobre determinadas linguagens como python e R e começar a manipular outros meios como Matlab e Latex, para estar o mais capacidado possivel para enfrentar qualquer problema real ligado a matemática computacional de forma clara e poder auxiliar possívelmente na extensão do conhecimento sobre a área através de pesquisas academicas, projetos científicos, meios de ensino entre outras diversas possibilidades que possam ser usadas.}
   
   \noindent
   \textsf{\_\_\_\_\_}
   
   \noindent
   \textsf{Atenciosamente}
   
   \noindent
   \textsf{Fabio Carvalho de Souza }
   
   \noindent
   \textsf{Aluno de Graduação em Matemática Aplicada e Computacional }
   
   \noindent
   \textsf{\_\_\_\_\_}
   
   \noindent
   \textsf{Referências das imagens usadas:}
   
   \noindent
   \textsf{Figura 1:Imagem disponivel em https://social.stoa.usp.br/articles/0042/1851/patrocinador-imeusp.png}
   
   \noindent    
   \textsf{Figura 2:Imagem disponivel em https://vivermelhor.info/codigo-emagrecer-de-vez/duvidas/}
   \end{indentation} 
   
  
  \end{document}